\documentclass[pdftex,12pt,oneside,a4paper,brazil]{abntex2}

\usepackage[brazil]{babel}
\usepackage[utf8]{inputenc}
\usepackage{amsmath}
\usepackage{amssymb}
\usepackage{amsthm}
\usepackage{indentfirst}
\usepackage{graphicx}
\usepackage{multicol,lipsum}
\usepackage{array}
\usepackage{enumitem}
\usepackage{float}
\usepackage[hang]{footmisc}
\usepackage{hyperref}
\usepackage{url}

\usepackage{eso-pic}
\usepackage{lipsum}
\usepackage{transparent}

\usepackage{imakeidx}
\makeindex[columns=3, title=Alphabetical Index, intoc]

\usepackage[
backend=biber,
style=abnt,
language=brazil
]{biblatex}
\addbibresource{referencias.bib}



\begin{document}

\pagenumbering{arabic}
    \begin{sloppypar}
    
    \selectlanguage{brazil}
    \frenchspacing
    	
        % Capa
        \begin{center}
    \LARGE Universidade Tiradentes\\
    \LARGE Ciência da Computação\\ 
    \vspace{15pt}
    \vspace{95pt}
\end{center}

\begin{center}
    \textbf{Pedro Henrique Moraes Silva Poderoso}\\
    \textbf{João Victor Sales de Santana Melo}\\
    \textbf{Davi Gonçalves Nascimento}\\
    \vspace{15pt}
    \vspace{95pt}
\end{center}

\begin{center}
    \vspace{1cm}
    \textbf{\large Campeonato Amador Fut-7}\\
    \textbf{\large Estrutura de Dados}\\
    \vspace{3.5cm} % decimal point, not comma
\end{center}

\vspace{1cm}

\begin{center}
    \vspace{\fill}
    Aracaju - SE \\
    2025
\end{center}

\newpage

        \cleardoublepage
        
        % Contra capa
        \begin{center}
    \textbf{Pedro Henrique Moraes Silva Poderoso}\\
    \textbf{João Victor Sales de Santana Melo}\\
    \textbf{Davi Gonçalves Nascimento}\\
    \vspace{15pt}
    \vspace{95pt}
\end{center}

\vspace{15pt}

\begin{center}
    \vspace{1cm}
    \textbf{\large Campeonato Amador Fut-7}\\
    \textbf{\large Estrutura de Dados}\\
    \vspace{3.5cm}
\end{center}

\vspace{1.5cm}

\begin{flushright}
\begin{list}{}{
    \setlength{\leftmargin}{8.5cm}
    \setlength{\rightmargin}{0cm}
    \setlength{\labelwidth}{0pt}
    \setlength{\labelsep}{\leftmargin}}
    \item Apresentação do projeto Campeonato Amador Fut-7 ,apresentada como requisito parcial da avaliação da disciplina \textit{Estrutura de Dados}, ministrada pelo Prof. \textit{Fausto Bernard}, no 2º semestre de 2025.
\end{list}
\end{flushright}

\vspace{1cm}

\begin{center}
    \vspace{\fill}
    Aracaju - SE \\
    2025
\end{center}
\newpage

        \cleardoublepage
        
        % Sumário
        \vspace{1cm}
        \tableofcontents*
        %\thispagestyle{empty}
        \cleardoublepage
		
        % Introdução
        \chapter[INTRODUÇÃO]{INTRODUÇÃO}\label{cap:refcomandos}
        \section{Projeto}
	Esse projeto é um gerenciador de campeonatos, é proposto a arquitetura e funcionamento de um sistema moderno de classificações, colocações e decisões de partidas, todas automáticas e com relatorios. Além disso a implementação de um ranking por jogador. Os métodos e etapas desse campeonato foram sintetizadas a partir das desctitas por \textcite{aubatuba2024}.

\section{Equipes}
	Cada equipe é constituida de 7 jogadores, formadas por uma lista duplamente encadeada, pela necessidade de um alocamento dinâmico e não indexação dos jogadores na estrutura.

\section{Ranqueamento de Jogadores}
	O sistema de ranking dos jogadores utiliza a quantidade de gols e assistências realizadas pelo jogador no campeonato como as estatísticas para a classificação final.

\section{Organização}
	O chaveamento das equipes inscritas considera fase de grupos, oitavas de final, quartas de final, semi final, final e disputa do 3º colocado, sendo o último a disputa entre as 2 equipes das semi finais que nao foram classificadas para a final
	
	\subsection{Ordem de Partidas}
		A organização desses confrontos serão realizados com filas, sendo que as quartas de final somente serão iniciadas quando não houver mais disputas pendentes nas oitavas, dessa forma até a final. Portanto é uma fila contendo as etapas do campeonato.
		Cada etapa é composta por confrontos entre 2 equipes, sendo uma fila, tal que um confronto somente é ralizado quando outro finalizar. O encadeamento do confronto é definido por uma intercalação entre os diferentes lados do campeonato, assumindo que esse sistema usa o modelo de campeonato ilustrado na Fig.~\ref{fig:figura1}.
		
		\begin{figure}[h!]
			\centering
			\includegraphics[scale=.13]{figuras/6740682-chave-de-torneio-de-equipe-vetor.jpg}
			\caption{estrutura utilizada para a logica do sistema}
			\label{fig:figura1}
		\end{figure}
		
		A fila de controntos é formada pelo intercalamento dos times de ambos os lados, como: primeiro da esquera, primeiro da direita, segundo da esquerta, segundo da direita, e assim sucessivamente.
		
		\subsection{Fase de grupos}
			Na fase de grupos terá no minimo 4 grupos (A, B, C, D, ...) de 4 times cada passando para a proxima etapa somente os 1º e 2º colocado de cada grupo. Os confrontos nas oitavas, sendo a proxima etapa, são formados pelo 1º colocado de um grupo com o 2º de outro, evitando o partidas repetidas e favorecendo a competitividade.		


        
        % Justificativa
        \chapter[JUSTIFICATIVA]{JUSTIFICATIVA}\label{cap:refcomandos}
        Essa ideia surge da necessidade de uma organização mais formal e estruturada de um campeonato interno. Famosas como "baba", "pelada", ou "futebol dos cria", partidas amistosas de futebol, na categoria de futsal ou campo ou society, são comuns em todo territorio nacional, esse projeto é formado como uma forma de organizar em casos de estruturas mais complexas e com mais jogadores que o habitual em qualquer categoria anteriormente citadas, esse projeto pode ser mais utilizado em torneios amadores de maior porte, conhecidos como futebol de "várzea", onde são formadas equipes maiores, tais quais utilizadas no projeto.

Inicialmente as estatisticas do sistema serão geradas automaticamente, com o propótsito de teste e validação, isso é, os gols por partida serão gerados a partir de um intervalo definido (0 a 4 gols por partida), e distribuidos pelos jogadores os gols e assistencias. Posteriormente podendo ser desenvolvida uma forma de inserção manual dos resultados em partidas reais.


        
        % Metolodogia
        \chapter[METODOLOGIA]{METODOLOGIA}\label{cap:refcomandos}
        A aplicação necessáriamente deve implementar as Listas e Filas ou Pilhas. Nesse caso foram usadas Filas, As Listas serviram como uma forma genérica de armazenar uma serie de dados, como é demonstrado na Figura~\ref{fig:figura2}, em algumas classes como Disputa, sendo composta principalmente por uma Lista de dois Times. As Filas foram usadas no contexto de ordenação das operações, em operações que precisam ser encadeadas em ordem, como foi usado no campeonato para que primeiro acontecam as Oitavas, em seguida Quartas e assim por diante, não sendo possivel a remoção por cherry picking, isso é, por id.

O sistema foi desenvolvido usando Java, algumas dependencias foram adicionadas, das principais temos:
\begin{itemize}
	\item \textbf{Lombok}:
	
	Esse foi usado para facilitar e agilizar a criação das classes, evidentemente torna mais legivel o código, visto que é mais simples e possui menos médotos as classes, principalmente aquelas com mais

	\item \textbf{JavaFaker}
	
	Esse é relevante o suficiente para ser tratado no seguinte tópico. Em suma ele gera dados.
	
\end{itemize}

\section{Geração de Dados}
Essa aplicação é um MPV, sendo incompleta para uso final. Ela apenas constitui da lógica e aplicação dos requisitos de forma necessária para uma apresentação inicial, visto também que foi feita em menos de 7 (sete) dias. Por isso, é impossível o usuário, também tenha em mente que não há sistema de login, digitar as informações de cada jogador, de cada time e os individuais resultados de cada disputa. Também é relevante destacar que são muitos dados a serem digitados nesse caso, contendo mais de 100 jogadores e 16 disputas nas oitavas, 8 nas quartas e assim por diante.

Visto esse grande numero de dados, preferimos manter apenas automático por enquanto, assim é gerado automaticamente a quantidade de times necessária para o campeonato, cada time com seus respectivos jogadores. Essa geração foi feira com o JavaFaker, contendo ferramentas para gerar nomes de time, empresas, pessoas e outros importantes campos. A geração dos resultados por disputa foi feita com a biblioteca Random do Java, para gerar os gols por cada time.

O processo de desenvolvimento seguiu etapas para sua conclusão e melhor estrutura, inicialmente foram implementadas as classes de dominio, como Jogador, Time, Disputa e outras de igual importância, em seguida a estrutura inicial dos menus, sem os métodos implementados, detalhe para que foi criado um sistema de menus, uma Classe, para padronizar os menus e reduzir os pontos de falha do sistema, como entrada de dados. Em seguida foram desenvolvidos os repositórios para armazenar os dados dos jogadores e times, logo após a logica do Campeonato. Essa lógica é mais complexa e está na própria classe Campeonato, que se comporta como estática, seus atributos e médotos. Em paralelo com o Campeonato foi implementado o menu de Relatórios, mostrando os resultados de cada fase e posições dos times, vencedor da final, segundo e terceiro colocado e placar de gols dos jogadores.

\begin{figure}[h!]
	\centering
	\includegraphics[scale=.14]{figuras/diagrama_classes.png}
	\caption{Diagrama de Classes da aplicação}
	\label{fig:figura2}
\end{figure}
		
    	\clearpage
    	\printbibliography
        
    \end{sloppypar}
\end{document}
% colocar o nº em cada página a partir da introdução no lado direito em cima
